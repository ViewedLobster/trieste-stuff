\documentclass{article}

\usepackage{amsmath, amssymb}
\usepackage[usenames,dvipsnames]{xcolor}

\newcommand{\rulename}{\textsc}
\newcommand{\rn}{\rulename}

\newcommand{\irule}[3]{\frac{\begin{array}{c}#2\end{array}}{\begin{array}{l}#3\end{array}}{~\small\mbox{(\rulename{#1})}}}

\newcommand{\labelFont}{\texttt}
\newcommand{\cursorColor}{LimeGreen}

\newcommand{\lbl}{\ensuremath{\mathit{lbl}}}
\newcommand{\lbls}{\ensuremath{\mathit{lbls}}}
\newcommand{\Top}{\ensuremath{\labelFont{Top}}}
\newcommand{\Group}{\ensuremath{\labelFont{Group}}}
\newcommand{\Invalid}{\ensuremath{\labelFont{Invalid}}}
\newcommand{\Unclosed}{\ensuremath{\labelFont{Unclosed}}}

\newcommand{\tm}{\ensuremath{\mathtt{tm}}}
\newcommand{\tms}{\ensuremath{\mathtt{tms}}}

\newcommand{\Tm}{\ensuremath{\check{\tm}}}
\newcommand{\Tms}{\ensuremath{\check{\tms}}}

\newcommand{\Tmc}{\ensuremath{\overset{ \textcolor{\cursorColor}{\bowtie}}{\tm}}}
\newcommand{\Tmcb}{\ensuremath{\overset{{\bowtie}}{\tm}}}
\newcommand{\Tmcs}{\ensuremath{\overset{\textcolor{\cursorColor}{\bowtie}}{\tms}}}

%% Actions
\newcommand{\action}{\ensuremath{\alpha}}
\newcommand{\add}[1]{\ensuremath{\texttt{add}~#1}}
\newcommand{\push}[1]{\ensuremath{\texttt{push}~#1}}
\newcommand{\pop}[1]{\ensuremath{\texttt{pop}~#1}}
\newcommand{\trypop}[1]{\ensuremath{\texttt{trypop}~#1}}
\newcommand{\seq}[1]{\ensuremath{\texttt{seq}~#1}}
\newcommand{\term}[1]{\ensuremath{\texttt{term}~#1}}
\newcommand{\invalid}{\ensuremath{\texttt{invalid}}}
\newcommand{\unclosed}{\ensuremath{\texttt{unclosed}}}
\newcommand{\done}{\ensuremath{\texttt{done}}}

%% Cursors and matching
\newcommand{\cursor}{\ensuremath{\downarrow}}
\newcommand{\mstart}{\ensuremath{\textcolor{\cursorColor}{\rhd}}}
\newcommand{\mend}{\ensuremath{\textcolor{\cursorColor}{\lhd}}}

\newcommand{\bindings}{\ensuremath{\mathcal{M}}}
\newcommand{\nobindings}{\ensuremath{\emptyset}}
\newcommand{\extend}[2]{\ensuremath{#1\overset{\rightarrow}{\cup}#2}}

\newcommand{\select}[1]{\mstart\colorbox{\cursorColor}{$#1$}\mend}

\newcommand{\steps}[1]{\ensuremath{\xrightarrow{#1}}}
\newcommand{\matches}[4]{\ensuremath{#1 \vdash #2 \Rightarrow #3 ~|~ #4}}
\newcommand{\xmatches}[4]{\ensuremath{
    \begin{array}{rl}
      #1 \vdash & #2 \Rightarrow\\
                & #3 ~|~ #4
    \end{array}}}

%% Patterns
\newcommand{\pseq}[2]{\ensuremath{#1 \cdot #2}}
\newcommand{\por}[2]{\ensuremath{#1/#2}}
\newcommand{\children}[2]{\ensuremath{#1} \Downarrow #2}
\newcommand{\pnot}[1]{\ensuremath{\lnot #1}}
\newcommand{\maybe}[1]{\ensuremath{#1?}}
\newcommand{\many}[1]{\ensuremath{#1^*}}
\newcommand{\any}{\ensuremath{\text{\underline{\hspace{0.6em}}}}}
\newcommand{\bind}[2]{\ensuremath{#1@#2}}
\newcommand{\paction}[2]{\ensuremath{#1\circ #2}}
\newcommand{\fb}[1]{\ensuremath{\rightarrow(#1)}}
\newcommand{\nfb}[1]{\ensuremath{\not\rightarrow(#1)}}
\newcommand{\pin}[1]{\ensuremath{\mathit{In}(#1)}}
\newcommand{\pstart}{\ensuremath{\mathit{Start}}}
\newcommand{\pend}{\ensuremath{\mathit{End}}}

\newcommand{\arrayheading}[2]{\multicolumn{#1}{l}{\mbox{#2}}}

\synctex=1

\begin{document}

\section{Parsing in Trieste}

\[
  \begin{array}{rcl}
    \arrayheading{3}{Labels}\\
    \lbl  & \in & \{\Top, \Group, \Invalid, \Unclosed\}\cup \mathcal{L} \\
    \lbls & ::= & \lbl_1 .. \lbl_n\\
    \\
    \arrayheading{3}{Terms}\\
    \tm  & ::= & (\lbl~\tms) \\
    \tms & ::= & \tm ~ \tms ~|~ \epsilon \\
    \\
    \arrayheading{3}{Terms with a single cursor}\\
    \Tm  & ::= & (\lbl~\Tms)\\
    \Tms & ::= & \tm ~ \Tms ~|~ \Tm ~|~ \cursor\\
    \\
    \arrayheading{3}{Contexts (shallow)}\\
    \Tm[\bullet] & ::= & (\lbl~\Tms[\bullet])\\
    \Tms[\bullet] & ::= & \tm ~ \Tms[\bullet] ~|~ \bullet\\
    \\
    \arrayheading{3}{Actions}\\
    \action & ::= & \add{\lbl}\\
            & ~|~ & \push{\lbl}\\
            & ~|~ & \pop{\lbl}\\
            & ~|~ & \pseq{\lbl}\\
            & ~|~ & \term{\lbls}\\
            & ~|~ & \trypop{\lbl}\\
            & ~|~ & \invalid\\
            & ~|~ & \unclosed\\
            & ~|~ & \done\\
  \end{array}
\]

A term \tm{} is an $n$-ary tree where each node has a label \lbl.
%
During parsing, terms \Tm{} have a single cursor \cursor{} in them,
always ``furthest to the right'' in the tree.
%
We use $\Tm[\bullet]$ to denote a term whose right-most child is a
hole $\bullet$ that can be filled with a sequence of terms.

We use the rule \rn{ctx} to ``zoom in'' close enough to the cursor
to apply a rule. Actions \term{} and \done{} are always applied to
the whole term and are therefore excluded from being applied in a
context:


\[
\irule{ctx}{
\Tm_1 \steps{\action} \Tm_2\\
\action \notin \{\term, \done\}
}{
\Tm[\Tm_1] \steps{\action} \Tm[\Tm_2]
}
\]

The action \add{\lbl} will add a \lbl-term to a group and
advance the cursor, creating a new group and moving inside it if
the cursor is not already inside one:

\[
  \begin{array}{c}
    \irule{add-in}{}{
    (\Group~\Tms[\cursor]) \steps{\add{\lbl}} (\Group~\Tms[(\lbl) \cursor])
    }
    \\\\

    \irule{add-create}{
    \lbl' \neq \Group
    }{
    (\lbl'~\Tms[\cursor]) \steps{\add{\lbl}} (\lbl'~\Tms[(\Group~(\lbl) \cursor)])
    }
  \end{array}
\]

Pushing a label adds a term (creating a group if needed) and moves
the cursor inside it:

\[
  \begin{array}{c}
    \irule{push-in}{}{
    (\Group~\Tms[\cursor]) \steps{\push{\lbl}} (\Group~\Tms[(\lbl~\cursor)])
    }
    \\\\

    \irule{push-create}{
    \lbl' \neq \Group
    }{
    (\lbl'~\Tms[\cursor]) \steps{\push{\lbl}} (\lbl'~\Tms[(\Group~(\lbl~\cursor))])
    }
  \end{array}
\]

The action \pop{\lbl} moves the cursor out of the innermost term,
assuming its label is \lbl. If the label does not match, an
error is reported (defined below):

\[
  \begin{array}{c}
    \irule{pop-ok}{
    (\lbl'~\Tms[(\lbl~\Tms'[\cursor])]) \steps{\trypop{\lbl}} \Tm
    }{
    (\lbl'~\Tms[(\lbl~\Tms'[\cursor])]) \steps{\pop{\lbl}} \Tm
    }
    \\\\

    \irule{pop-fail}{
    \lbl'' \neq \lbl\\
    (\lbl''~\Tms'[\cursor]) \steps{\invalid} \Tm
    }{
    (\lbl'~\Tms[(\lbl''~\Tms'[\cursor])]) \steps{\pop{\lbl}}
    (\lbl'~\Tms[\Tm])
    }
  \end{array}
\]

Popping is done via an auxiliary action \trypop{\lbl} which does
nothing on a label mismatch:

\[
  \begin{array}{c}
    \irule{trypop-ok}{}{
    (\lbl'~\Tms[(\lbl~\Tms'[\cursor])]) \steps{\trypop{\lbl}}
    (\lbl'~\Tms[(\lbl~\Tms'[\epsilon])~\cursor])
    }
    \\\\
    \irule{trypop-fail}{
    \lbl'' \neq \lbl
    }{
    (\lbl'~\Tms[(\lbl''~\Tms'[\cursor])]) \steps{\trypop{\lbl}}
    (\lbl'~\Tms[(\lbl''~\Tms'[\cursor])])
    }
  \end{array}
\]

Errors are reported by adding a term with the label \Invalid. If
the cursor is already next to an \Invalid-term, the extent of
that error is extended (currently just modeled as a no-op) rather
than adding another \Invalid-term:

\[
  \begin{array}{c}
    \irule{invalid-extend}{}{
    (\lbl~\Tms[(\Invalid)~\cursor]) \steps{\invalid}
    (\lbl~\Tms[(\Invalid)~\cursor])
    }
    \\\\

    \irule{invalid-empty}{
    (\lbl~\cursor) \steps{\add{\Invalid}} \Tm
    }{
    (\lbl~\cursor) \steps{\invalid} \Tm
    }
    \\\\

    \irule{invalid-non-empty}{
    \lbl' \neq \Invalid\\
    (\lbl~\Tms[(\lbl'~\Tms') \cursor]) \steps{\add{\Invalid}} \Tm
    }{
    (\lbl~\Tms[(\lbl'~\Tms') \cursor]) \steps{\invalid} \Tm
    }
  \end{array}
\]

The action \seq{\lbl} finishes the current group and makes it a
child of a \lbl-term, creating that term if it is not the parent
of the current group. Performing the \seq{}-action when not in a
group is an error:

\[
  \begin{array}{c}
    \irule{seq-in}{}{
    (\lbl~\Tms[(\Group~\Tms'[\cursor])]) \steps{\seq{\lbl}}
    (\lbl~\Tms[(\Group~\Tms'[\epsilon])~\cursor])
    }\\\\
    \irule{seq-create}{
    \lbl' \neq \lbl
    }{
    (\lbl'~\Tms[(\Group~\Tms'[\cursor])]) \steps{\seq{\lbl}}
    (\lbl'~\Tms[(\lbl~(\Group~\Tms'[\epsilon])~\cursor)])
    }\\\\
    \irule{seq-fail}{
    \lbl' \neq \Group\\
    (\lbl'~\Tms[\cursor]) \steps{\invalid} \Tm{}
    }{
    (\lbl'~\Tms[\cursor]) \steps{\seq{\lbl}} \Tm{}
    }\\\\
  \end{array}
\]

The action \term{\lbls} finishes the current group (if any) and
tries to move out of enclosing terms in the order of the labels in
\lbls, ignoring any non-matching labels. Note that this rule may
use the \rn{ctx} rule to find the right level for popping.

\[
  \begin{array}{c}
    \irule{term}{
    \Tm \steps{\trypop{\Group}} \Tm_0\\
    \forall~i\in[1..n]. \Tm_{i-1} \steps{\trypop{\lbl_i}} \Tm_i
%    \lbl_2(\Tms_2[\lbl_3(\Tms_3[\cursor])]) \steps{\trypop{\Group}} \Tm\\
%    \lbl_1(\Tms_1[\Tm]) \steps{\trypop{\lbl}} \Tm'
    }{
    \Tm \steps{\term{\lbl_1..\lbl_n}} \Tm_n
%    \lbl_1(\Tms_1[\lbl_2(\Tms_2[\lbl_3(\Tms_3[\cursor])])]) \steps{\term{\lbl}}
%    \Tm'
    }
  \end{array}
\]

Finally, the \done{} action finishes the final group (if any) and
closes the top-most term, which must have label \Top{} (As an aside,
note that we can't use \trypop{\Group} as \trypop{} assumes that
the cursor is at least at depth 2, which is not true when the
cursor a direct child of the top term). Note that the resulting
term is a term without a cursor.

\[
  \begin{array}{c}
    \irule{done}{}{
    (\Top~\Tms[\cursor]) \steps{\done} (\Top~\Tms[\epsilon])
    }
    \\\\
    \irule{done-group}{}{
    (\Top~\Tms[(\Group~\Tms'[\cursor])]) \steps{\done} (\Top~\Tms[(\Group~\Tms'[\epsilon])])
    }
  \end{array}
\]

If the cursor is \emph{not} a direct child of the top node (or in
a group that is a direct child of the top node), parsing has
terminated prematurely and we add \Unclosed{} terms along the path
to the top until we can finish the term:

\[
  \begin{array}{c}
    \irule{done-unclosed}{
    \lbl \neq \Group\\
    (\Top~\Tms[(\lbl~\Tms')]) \steps{\unclosed} \tm\\
    }{
    (\Top~\Tms[(\lbl~\Tms')]) \steps{\done} \tm
    }
    \\\\
    \irule{unclosed-top}{}{
    (\Top~\Tms[\cursor]) \steps{\unclosed} \Top(\Tms[(\Unclosed)])
    }
    \\\\
    \irule{unclosed-step}{
    \Tm \steps{\unclosed} \Tm'\\
    \Tm' \steps{\unclosed} \tm
    }{
    \Tm \steps{\unclosed} \tm
    }
    \\\\
    \irule{unclosed-add}{
    \lbl' \neq \Group
    }{
    (\lbl~\Tms[(\lbl'~\Tms'[\cursor])]) \steps{\unclosed}\\
    (\lbl~\Tms[(\lbl'~\Tms'[(\Unclosed)]) \cursor])
    }
    \\\\
    \irule{unclosed-group}{}{
    (\lbl~\Tms[(\lbl'~\Tms'[(\Group~\Tms''[\cursor])])]) \steps{\unclosed}\\
    (\lbl~\Tms[(\lbl'~\Tms'[(\Group~\Tms''[\epsilon]) ~ (\Unclosed)]) \cursor])
    }
  \end{array}
\]

\section{Pattern Matching}

\[
  \begin{array}{rcl}
    \arrayheading{3}{Terms with a sequence being matched}\\
    \Tmc & ::= & (\lbl~\Tmcs) ~|~ (\lbl~\tms~\select{\tms}~\tms)\\
    \Tmcs & ::= & \Tmc ~ \tms ~|~ \tm ~ \Tmcs\\
    \\
    \arrayheading{3}{Contexts (shallow)}\\
    \Tmc[\bullet] & ::= & (\lbl~\Tmcs[\bullet])\\
    \Tmcs[\bullet] & ::= & \bullet ~ \tms ~|~ \tm ~ \Tmcs[\bullet]\\
    \\
    \arrayheading{3}{Bindings}\\
    x, y      & \in & \mathcal{X}\\
    \bindings & \in & \mathcal{X} \mapsto \tms\\
    \\
    \multicolumn{3}{l}{
    \begin{array}{crll}
      \arrayheading{4}{Patterns}\\
      p & ::= & \lbl & \text{a term with label $\lbl$}\\
        & ~|~ & \pseq{p}{p} & \text{sequence}\\
        & ~|~ & \por{p}{p} & \text{choice}\\
        & ~|~ & \children{p}{p} & \text{with children}\\
        & ~|~ & \pnot{p} & \text{negation}\\
        & ~|~ & \maybe{p} & \text{option}\\
        & ~|~ & \many{p} & \text{repetition}\\
        & ~|~ & \any{} & \text{any term}\\
        & ~|~ & \bind{p}{x} & \text{bind}\\
        & ~|~ & \paction{p}{f} & \text{action}\\
        & ~|~ & \fb{p} & \text{followed by}\\
        & ~|~ & \nfb{p} & \text{not followed by}\\
        & ~|~ & \pin{\lbl} & \text{appears as child of}\\
        & ~|~ & \pstart{} & \text{before all children}\\
        & ~|~ & \pend{} & \text{after all children}\\
    \end{array}
    }
  \end{array}
\]

During matching we are selecting a sequence of sibling terms. A
term \Tmc{} is a term where one such sequence has been selected.
We represent the subterm whose children are being matched as
$(\lbl~\tms_{pre}~\select{\tms}~\tms_{post})$: a $\lbl$ term
where some subsequence $\tms$ of its children has been selected,
$\tms_{pre}$ being the terms before the match and $\tms_{post}$
the terms after the match. Matching commences by looking at terms
in $\tms_{post}$ and moving them into the matched sequence, or
equivalently, advancing the cursor $\mend$. Note that the coloring
has no semantic meaning.

The result of matching, apart from moving the cursor $\mend$, is a
set of bindings $\bindings$ which maps variables to sequences of
terms. If the same name is used more than once, the later binding
is preferred. We write $\extend{\bindings_1}{\bindings_2}$ to mean
the union of $\bindings_1$ and $\bindings_2$ where bindings from
$\bindings_2$ override overlapping bindings from $\bindings_1$:

\[
    \extend{\bindings_1}{\bindings_2} = \bindings_2\cup \{x\mapsto \bindings_1(x) ~|~ x \in (\mathbf{dom}(\bindings_1)\setminus\mathbf{dom}(\bindings_2))\}
\]

The pattern matching rule takes the shape
$\matches{p}{\Tmc_1}{\Tmc_2}{\bindings}$, where we start from
$\Tmc_1$ and match $p$, (possibly) extending the selection
resulting in $\Tmc_2$ and the bindings $\bindings$.

Throughout, we use $\lbl_\pi$ as the label of the parent of the
nodes we are matching on.
%
Patterns $\bind{p}{x}$ which introduce a binding remembers the
terms that were matched in the resulting binding set:

\[
  \begin{array}{c}
    \irule{match-bind}{
    \matches{p}{(\lbl_\pi~\tms_{pre}~\select{\tms}~\tms_{post})}
               {(\lbl_\pi~\tms_{pre}~\select{\tms~\tms'}~\tms'_{post})}{\bindings}
    }{
    \xmatches{\bind{p}{x}}
    {(\lbl_\pi~\tms_{pre}~\select{\tms}~\tms_{post})}
    {(\lbl_\pi~\tms_{pre}~\select{\tms~\tms'}~\tms'_{post})}{\extend{\bindings} \{x\mapsto\tms'\}}
    }
  \end{array}
\]

% Patterns match on labels $\lbl$ (with $\any$ matching any label),
% and can be combined in several ways:

% \begin{itemize}
% \item $\pseq{p_1}{p_2}$     --- $p_1$ followed by $p_2$
% \item $\por{p_1}{p_2}$      --- $p_1$ or $p_2$
% \item $\children{p_1}{p_2}$ --- $p_1$ with children matching $p_2$
% \item $\pnot{p}$            --- not $p$
% \item $\maybe{p}$           --- optionally $p$
% \item $\many{p}$            --- zero or more $p$
% \end{itemize}

% We can bind the result of matching a subpattern to a name with
% the pattern $\bind{p}{x}$. There are also a number of patterns
% which do not advance the cursor:

% \begin{itemize}
% \item $\fb{p}$     --- followed by $p$
% \item $\nfb{p}$    --- not followed by $p$
% \item $\pin{\lbl}$ --- inside a node with label $\lbl$
% \item $\pstart{}$  --- before all children of the current term
% \item $\pend{}$    --- after all children of the current term
% \end{itemize}

\pstart{} and \pend{} match the very start and the very end
respectively of the full sequence of children:

\[
  \begin{array}{c}
    \irule{match-start}{}{
    \matches{\pstart}
    {(\lbl_\pi~\mstart\mend~\tms_{post})}
    {(\lbl_\pi~\mstart\mend~\tms_{post})}{\nobindings}
    }
    \\\\
    \irule{match-end}{}{
    \xmatches{\pend}
    {(\lbl_\pi~\tms_{pre}~\select{\tms})}
    {(\lbl_\pi~\tms_{pre}~\select{\tms})}{\nobindings}
    }
  \end{array}
\]

$\pin{\lbl}$ succeeds if we are matching directly inside a $\lbl$
term:

\[
  \begin{array}{c}
    \irule{match-in}{}{
    \xmatches{\pin{\lbl_\pi}}
    {(\lbl_\pi~\tms_{pre}~\select{\tms}~\tms_{post})}
    {(\lbl_\pi~\tms_{pre}~\select{\tms}~\tms_{post})}{\nobindings}
    }
  \end{array}
\]

The lookahead patterns $\fb{p}$ and $\nfb{p}$ tries the pattern
$p$, but does not advance the cursor nor remember bindings.

\[
  \begin{array}{c}
    \irule{match-followed-by}{
    \matches{p}{(\lbl_\pi~\tms_{pre}~\select{\tms}~\tms_{post})}{\Tmc}{\bindings}
    }{
    \xmatches{\fb{p}}
    {(\lbl_\pi~\tms_{pre}~\select{\tms}~\tms_{post})}
    {(\lbl_\pi~\tms_{pre}~\select{\tms}~\tms_{post})}{\nobindings}
    }
    \\\\
    \irule{match-not-followed-by}{
    \lnot(\matches{p}{(\lbl_\pi~\tms_{pre}~\select{\tms}~\tms_{post})}{\Tmc}){\bindings}
    }{
    \xmatches{\nfb{p}}
    {(\lbl_\pi~\tms_{pre}~\select{\tms}~\tms_{post})}
    {(\lbl_\pi~\tms_{pre}~\select{\tms}~\tms_{post})}{\nobindings}
    }
  \end{array}
\]

Matching on a label $\lbl$ succeeds if it is the next label.
Matching on a wild-card always succeeds as long as there is a term
to match on:

\[
  \begin{array}{c}
    \irule{match-label}{}{
    \xmatches{\lbl}
    {(\lbl_\pi~\tms_{pre}~\select{\tms}~(\lbl~\tms')~\tms_{post})}
    {(\lbl_\pi~\tms_{pre}~\select{\tms~(\lbl~\tms')}~\tms_{post})}{\nobindings}
    }
    \\\\
    \irule{match-any}{}{
    \xmatches{\any}
    {(\lbl_\pi~\tms_{pre}~\select{\tms}~(\lbl~\tms')~\tms_{post})}
    {(\lbl_\pi~\tms_{pre}~\select{\tms~(\lbl~\tms')}~\tms_{post})}{\nobindings}
    }
  \end{array}
\]

Patterns in sequence are matched left to right, with later
bindings being preferred over earlier:

\[
  \begin{array}{c}
    \irule{match-sequence}{
    \matches{p_1}{(\lbl_\pi~\tms_{pre}~\select{\tms}~\tms_{post})}{\Tmc_1}{\bindings_1}\\
    \matches{p_2}{\Tmc_1}{\Tmc_2}{\bindings_2}\\
    }{
    \matches{\pseq{p_1}{p_2}}
    {(\lbl_\pi~\tms_{pre}~\select{\tms}~\tms_{post})}
    {\Tmc_2}{\extend{\bindings_1}{\bindings_2}}
    }
  \end{array}
\]

Alternative patterns $\por{p_1}{p_2}$ picks the first of the two
patterns that matches [Could not be zoomed in?]:

\[
  \begin{array}{c}
    \irule{match-or-l}{
    \matches{p_1}{(\lbl_\pi~\tms_{pre}~\select{\tms}~\tms_{post})}{\Tmc}{\bindings}
    }{
    \matches{\por{p_1}{p_2}}
    {(\lbl_\pi~\tms_{pre}~\select{\tms}~\tms_{post})}
    {\Tmc}{\bindings}
    }
    \\\\
    \irule{match-or-r}{
    \lnot(\matches{p_1}{(\lbl_\pi~\tms_{pre}~\select{\tms}~\tms_{post})}{\Tmc_1}{\bindings_1})\\
    \matches{p_2}{(\lbl_\pi~\tms_{pre}~\select{\tms}~\tms_{post})}{\Tmc_2}{\bindings_2}
    }{
    \matches{\por{p_1}{p_2}}
    {(\lbl_\pi~\tms_{pre}~\select{\tms}~\tms_{post})}
    {\Tmc_2}{\bindings_2}
    }
  \end{array}
\]

The rule for matching children $\children{p_1}{p_2}$ is probably
the most complicated. First, the pattern $p_1$ matches normally
and defines the movement of the cursor to include $\tms_1$.
Additionally, we extract the first term $(\lbl~\tms_{11})$ inside
which we match $p_2$. We transfer the bindings from the inner
match to the final results, preferring these bindings to ones from
the outer match.
%
Note that $p_1$ must match at least one term.

\[
  \begin{array}{c}
    \irule{match-children}{
    \matches{p_1}{(\lbl_\pi~\tms_{pre}~\select{\tms}~\tms_{post})}
                 {(\lbl_\pi~\tms_{pre}~\select{\tms~\tms_1}~\tms'_{post})}{\bindings_1}\\
    \tms_1 = (\lbl~\tms_{11})~\tms'_1\\
    \matches{p_2}{(\lbl~\mstart\mend~\tms_{11})}{\Tmc}{\bindings_2}
    }{
    \xmatches{\children{p_1}{p_2}}
    {(\lbl_\pi~\tms_{pre}~\select{\tms}~\tms_{post})}
    {(\lbl_\pi~\tms_{pre}~\select{\tms~\tms_1}~\tms'_{post})}
    {\extend{\bindings_1}{\bindings_2}}
    }
  \end{array}
\]

The negation of a pattern $\pnot{p}$ matches the subsequent term
if $p$ does \emph{not} match. Note that this means that the next
term is consumed regardless of whether the pattern $p$ consumes
zero or more than one term:

\[
  \begin{array}{c}
    \irule{match-not}{
    \lnot(\matches{p}{(\lbl_\pi~\tms_{pre}~\select{\tms}~(\lbl~\tms')~\tms_{post})}{\Tmc}{\bindings})
    }{
    \xmatches{\pnot{p}}
    {(\lbl_\pi~\tms_{pre}~\select{\tms}~(\lbl~\tms')~\tms_{post})}
    {(\lbl_\pi~\tms_{pre}~\select{\tms~(\lbl~\tms')}~\tms_{post})}{\nobindings}
    }
  \end{array}
\]

Optional matches $\maybe{p}$ always succeeds, but matches $p$ if
possible:

\[
  \begin{array}{c}
    \irule{match-option-yes}{
    \matches{p}{(\lbl_\pi~\tms_{pre}~\select{\tms}~\tms_{post})}{\Tmc}{\bindings}
    }{
    \matches{\maybe{p}}
    {(\lbl_\pi~\tms_{pre}~\select{\tms}~\tms_{post})}
    {\Tmc}{\bindings}
    }
    \\
    \\
    \irule{match-option-no}{
    \lnot(\matches{p}{(\lbl_\pi~\tms_{pre}~\select{\tms}~\tms_{post})}{\Tmc}{\bindings})
    }{
    \xmatches{\maybe{p}}
    {(\lbl_\pi~\tms_{pre}~\select{\tms}~\tms_{post})}
    {(\lbl_\pi~\tms_{pre}~\select{\tms}~\tms_{post})}{\nobindings}
    }
  \end{array}
\]

Repetition of a pattern $\many{p}$ will greedily match $p$ as many
times as possible, preferring later bindings to earlier:

\[
  \begin{array}{c}
    \irule{match-many-more}{
    \matches{p}{(\lbl_\pi~\tms_{pre}~\select{\tms}~\tms_{post})}{\Tmc_1}{\bindings_1}\\
    \matches{\many{p}}{\Tmc_1}{\Tmc_2}{\bindings_2}
    }{
    \matches{\many{p}}
    {(\lbl_\pi~\tms_{pre}~\select{\tms}~\tms_{post})}
    {\Tmc_2}{\extend{\bindings_1}{\bindings_2}}
    }
    \\\\
    \irule{match-many-done}{
    \lnot(\matches{p}{(\lbl_\pi~\tms_{pre}~\select{\tms}~\tms_{post})}{\Tmc}{\bindings})
    }{
    \xmatches{\many{p}}
    {(\lbl_\pi~\tms_{pre}~\select{\tms}~\tms_{post})}
    {(\lbl_\pi~\tms_{pre}~\select{\tms}~\tms_{post})}{\nobindings}
    }
  \end{array}
\]

An action pattern $\paction{p}{f}$ comes equipped with a function
$f : \tms \mapsto \mathbf{bool}$ which must return $\mathbf{true}$
for the matched pattern:

\[
  \begin{array}{c}
    \irule{match-action}{
    \matches{p}{(\lbl_\pi~\tms_{pre}~\select{\tms}~\tms_{post})}
               {(\lbl_\pi~\tms_{pre}~\select{\tms~\tms'}~\tms'_{post})}{\bindings}\\
    f(\tms') = \mathbf{true}
    }{
    \xmatches{\paction{p}{f}}
    {(\lbl_\pi~\tms_{pre}~\select{\tms}~\tms_{post})}
    {(\lbl_\pi~\tms_{pre}~\select{\tms~\tms'}~\tms'_{post})}{\bindings}
    }
  \end{array}
\]


As with parsing, we use contexts to ``zoom in'' until we can apply
a matching rule:

\[
  \begin{array}{c}
    \irule{match-ctx}{
    \matches{p}{\Tmc_1}{\Tmc_2}{\bindings}
    }{
    \matches{p}{\Tmc[\Tmc_1]}{\Tmc[\Tmc_2]}{\bindings}
    }
  \end{array}
\]

\section{Term Rewriting}



\section*{TODO}

\begin{itemize}
\item Top-down vs. Bottom-up
\item Regex matching (requires carrying data)
\item Well-formedness
\item Symbol tables?
\end{itemize}

\section*{Quiz}

\begin{verbatim}
((T(Foo) * T(Bar)) << Any[Id]) >>
  [](Match& _) { return _(Id); }

(foo a) (bar b) --> a
\end{verbatim}

\begin{verbatim}
(In(Foo) << Any[Id]) >>
  [](Match& _) { return _(Id); }

(foo a) (bar b) --> Segmentation fault
\end{verbatim}

\begin{verbatim}
((!In(Bar)) * T(Baz)) >>
  [](Match& _) { return NodeDef::create(Fr0b); }

(foo baz) --> fr0b
\end{verbatim}

\end{document}