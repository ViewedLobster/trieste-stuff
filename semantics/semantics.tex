\documentclass{article}

\usepackage{amsmath}

\newcommand{\typerule}[3]{\frac{\begin{array}{c}#2\end{array}}{\begin{array}{c}#3\end{array}}{~\small\mbox{(\textsc{#1})}}}

\newcommand{\labelFont}{\texttt}

\newcommand{\lbl}{\ensuremath{\mathit{lbl}}}
\newcommand{\Group}{\ensuremath{\labelFont{Group}}}
\newcommand{\Invalid}{\ensuremath{\labelFont{Invalid}}}
\newcommand{\Unclosed}{\ensuremath{\labelFont{Unclosed}}}

\newcommand{\tm}{\ensuremath{\mathtt{tm}}}
\newcommand{\tms}{\ensuremath{\mathtt{tms}}}

\newcommand{\Tm}{\ensuremath{\check{\tm}}}
\newcommand{\Tms}{\ensuremath{\check{\tms}}}

\newcommand{\action}{\ensuremath{\alpha}}
\newcommand{\add}[1]{\ensuremath{\texttt{add}~#1}}
\newcommand{\push}[1]{\ensuremath{\texttt{push}~#1}}
\newcommand{\pop}[1]{\ensuremath{\texttt{pop}~#1}}
\newcommand{\seq}[1]{\ensuremath{\texttt{seq}~#1}}
\newcommand{\term}[1]{\ensuremath{\texttt{term}~#1}}
\newcommand{\invalid}{\ensuremath{\texttt{invalid}}}
\newcommand{\done}{\ensuremath{\texttt{done}}}

\newcommand{\cursor}{\ensuremath{\downarrow}}

\newcommand{\steps}[1]{\ensuremath{\xrightarrow{#1}}}

\newcommand{\arrayheading}[2]{\multicolumn{#1}{l}{\mbox{#2}}}

\begin{document}
\[
  \begin{array}{rcl}
    \arrayheading{3}{Labels}\\
    \lbl & \in & \{\Group, \Invalid, \Unclosed\}\cup \mathcal{L} \\
    \\
    \arrayheading{3}{Terms}\\
    \tm  & ::= & \lbl(\tms) \\
    \tms & ::= & \tm ~ \tms ~|~ \epsilon \\
    \\
    \arrayheading{3}{Terms with a single cursor}\\
    \Tm  & ::= & \lbl(\Tms)\\
    \Tms & ::= & \tm ~ \Tms ~|~ \Tm ~|~ \cursor\\
    \\
    \arrayheading{3}{Contexts (shallow)}\\
    \Tm[\bullet] & ::= & \lbl(\Tms[\bullet])\\
    \Tms[\bullet] & ::= & \tm ~ \Tms[\bullet] ~|~ \bullet\\
    \\
    \arrayheading{3}{Actions}\\
    \action & ::= & \add{\lbl}\\
            & ~|~ & \push{\lbl}\\
            & ~|~ & \pop{\lbl}\\
            & ~|~ & \seq{\lbl}\\
            & ~|~ & \term{\lbl}\\
            & ~|~ & \invalid\\
            & ~|~ & \done\\
  \end{array}
\]

A term \tm{} is an $n$-ary tree where each node has a label \lbl{}.
%
During parsing, terms \Tm{} have a single cursor \cursor{} in them,
always ``furthest to the right'' in the tree.
%
We use $\Tm[\bullet]$ to denote a term whose right-most child is a
hole $\bullet$ that can be filled with a sequence of terms.

We use the rule

\[
\typerule{ctx}{
\Tm_1 \steps{\action} \Tm_2\\
\action \notin \{\invalid, \done\}
}{
\Tm[\Tm_1] \steps{\action} \Tm[\Tm_2]
}
\]

\[
  \begin{array}{c}
    \typerule{add-in}{}{
    \Group(\Tms[\cursor]) \steps{\add{\lbl}} \Group(\Tms[lbl(\epsilon) \cursor])
    }\\\\
    \typerule{add-create}{
    \lbl' \neq \Group
    }{
    \lbl'(\Tms[\cursor]) \steps{\add{\lbl}} \lbl'(\Tms[\lbl(\epsilon) \cursor])
    }
  \end{array}
\]

\[
  \begin{array}{c}
    \typerule{push-in}{}{
    \Group(\Tms[\cursor]) \steps{\push{\lbl}} \Group(\Tms[\lbl(\cursor)])
    }\\\\
    \typerule{push-create}{
    \lbl' \neq \Group
    }{
    \lbl'(\Tms[\cursor]) \steps{\push{\lbl}} \lbl'(\Tms[\Group(\lbl(\cursor))])
    }\\\\
  \end{array}
\]

\[
  \begin{array}{c}
    \typerule{pop-ok}{}{
    \lbl'(\Tms[\lbl(\Tms'[\cursor])]) \steps{\pop{\lbl}}
    \lbl'(\Tms[\lbl(\Tms'[\epsilon])~\cursor])
    }\\\\
    \typerule{pop-fail}{
    \lbl'' \neq \lbl
    }{
    \lbl'(\Tms[\lbl''(\Tms'[\cursor])]) \steps{\pop{\lbl}}
    \lbl'(\Tms[\lbl''(\Tms'[\Invalid~\cursor])])
    }\\\\
  \end{array}
\]

\[
  \begin{array}{c}
    \typerule{seq-in}{}{
    \lbl(\Tms[\Group(\Tms'[\cursor])]) \steps{\seq{\lbl}}
    \lbl(\Tms[\Group(\Tms'[\epsilon])~\cursor])
    }\\\\
    \typerule{seq-create}{
    \lbl' \neq \lbl
    }{
    \lbl'(\Tms[\Group(\Tms'[\cursor])]) \steps{\seq{\lbl}}
    \lbl'(\Tms[\lbl(\Group(\Tms'[\epsilon])~\cursor)])
    }\\\\
    \typerule{seq-fail}{
    \lbl' \neq \Group
    }{
    \lbl'(\Tms[\cursor]) \steps{\seq{\lbl}}
    \lbl'(\Tms[\Invalid~\cursor])
    }\\\\
  \end{array}
\]

\end{document}